\documentclass[a4paper, 11pt, titlepage]{article}
\usepackage[utf8]{inputenc}
\usepackage{kvoptions-patch}
\usepackage[titulo={Práctica 1 - Entrega 2: Navegación local sin y con mapa}]{estilo}

\begin{document}
  \maketitle
  \tableofcontents
  \newpage

  \section{Resumen}
  Esta memoria describe brevemente el trabajo realizado para la entrega 2 de la práctica 1.

  Se han desarrollado únicamente las tareas de la navegación local sin mapa:
  \begin{itemize}
      \item Se ha aumentado el campo de visión del escáner láser del robot analizando el código y modificándolo convenientemente.
      \item Se ha implementado una técnica para evitar la detención del robot cerca del objetivo controlando el módulo del vector atractivo.
      \item Se ha corregido el comportamiento suicida del robot controlando el módulo de la velocidad lineal.
      \item Se ha diseñado una técnica de mínimos locales con aplazamiento de objetivos y control local de la posición.
      \item Se ha reducido el comportamiento oscilatorio del robot con técnicas aritméticas que suavizan el movimiento.
  \end{itemize}

  La siguiente sección describe con detalle cada una de estas técnicas.

  \section{Ejercicio 1}

  \subsection{Aumento del campo de visión}
  Para el aumento del campo de visión se analizó primero la definición del escáner láser en el archivo \emph{.world} que describe al robot.

  Nos aseguramos entonces de que el campo \emph{fov} de la definición \emph{sensor} fuera al menos 270.

  Sin embargo, esto define únicamente el rango físico del láser, no los datos que de él consideramos. Así, recalculamos las constantes \emph{MIN_SCAN_ANGLE_RAD} y \emph{MIN_SCAN_ANGLE_RAD} definidas en \emph{myPlannerLite.h} para que reflejaran nuestras necesidades: definimos la primera  a -135 grados y la segunda a 135, teniendo así un total de 270 grados.

  \subsection{Detención más rápida cerca del objetivo}
    Para que el robot no pare demasiado lento sobre el punto objetivo, acotamos la mínima velocidad que puede tener el robot, ya que al estar muy cerca del objetivo, el vector de movimiento resultante tiene módulo muy pequeño y por tanto la velocidad del robot es muy baja. Acotando por abajo dicha velocidad conseguimos que el robot no deje nunca de moverse a un ritmo aceptable.

  \subsection{Corrección del comportamiento suicida}
    Para solucionar este comportamiento simplemente hemos acotado la velocidad máxima del robot. Así, aunque el módulo del vector de movimiento resultante (debido a un factor de repulsión muy grande) fuera muy grande, al estar acotado daría tiempo al robot para girar y evitar el obstáculo.

  \subsection{Salida de mínimos locales}
    Lo primero ha sido determinar la siguiente cuestión: ¿Cómo sabemos que estamos atascados en un mínimo local? Nosotros lo determinamos si no salimos de un radio de 0.5 unidades de distancia durante 5 segundos (25 iteraciones a 5Hz). Una vez hemos decidido, mandamos al robot un goal en una posición aleatoria del mapa. Esto lo mantenemos durante 10 segundos, que es un tiempo viable para salir de una esquina, y después de eso volvemos a mandarle el goal original.

  \subsection{Reducción de oscilaciones}
    Para reducir las oscilaciones hemos ideado lo siguiente: como es la componente repulsiva la que las causa debido a las variaciones bruscas de la dirección del vector, nosotros le aplicamos la media a las dos últimas componentes repulsivas calculadas. Así se reduce la diferencia entre las iteraciones anteriores y la actual, evitando oscilaciones.
\end{document}
