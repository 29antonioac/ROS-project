\documentclass[a4paper, 11pt, titlepage]{article}
\usepackage[utf8]{inputenc}
\usepackage{kvoptions-patch}
\usepackage[titulo={Práctica 1 - Entrega 2: Navegación local sin y con mapa}]{estilo}

\begin{document}
  \maketitle
  \tableofcontents
  \newpage

  \section{Resumen}
  Lorem ipsum dolor sit amet, consectetur adipisicing elit, sed do eiusmod tempor incididunt ut labore et dolore magna aliqua. Ut enim ad minim veniam, quis nostrud exercitation ullamco laboris nisi ut aliquip ex ea commodo consequat. Duis aute irure dolor in reprehenderit in voluptate velit esse cillum dolore eu fugiat nulla pariatur. Excepteur sint occaecat cupidatat non proident, sunt in culpa qui officia deserunt mollit anim id est laborum.
  Lorem ipsum dolor sit amet, consectetur adipisicing elit, sed do eiusmod tempor incididunt ut labore et dolore magna aliqua. Ut enim ad minim veniam, quis nostrud exercitation ullamco laboris nisi ut aliquip ex ea commodo consequat. Duis aute irure dolor in reprehenderit in voluptate velit esse cillum dolore eu fugiat nulla pariatur. Excepteur sint occaecat cupidatat non proident, sunt in culpa qui officia deserunt mollit anim id est laborum.

  \section{Ejercicio 1}

  \subsection{Aumento del campo de visión}

  \subsection{Detención más rápida cerca del objetivo}

  \subsection{Corrección del comportamiento suicida}

  \subsection{Salida de mínimos locales}

  \subsection{Reducción de oscilaciones}
    Para reducir las oscilaciones hemos ideado lo siguiente: como es la componente repulsiva la que las causa debido a las variaciones bruscas de la dirección del vector, nosotros le aplicamos la media a las dos últimas componentes repulsivas calculadas. Así se reduce la diferencia entre las iteraciones anteriores y la actual, evitando oscilaciones.
\end{document}
