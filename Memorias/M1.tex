\documentclass[a4paper, 11pt, titlepage]{article}
\usepackage[utf8]{inputenc}
\usepackage{kvoptions-patch}
\usepackage[titulo={Práctica 1 - Entrega 2: Navegación local sin y con mapa}]{estilo}

\begin{document}
  \maketitle
  \tableofcontents
  \newpage

  \section{Resumen}
  Lorem ipsum dolor sit amet, consectetur adipisicing elit, sed do eiusmod tempor incididunt ut labore et dolore magna aliqua. Ut enim ad minim veniam, quis nostrud exercitation ullamco laboris nisi ut aliquip ex ea commodo consequat. Duis aute irure dolor in reprehenderit in voluptate velit esse cillum dolore eu fugiat nulla pariatur. Excepteur sint occaecat cupidatat non proident, sunt in culpa qui officia deserunt mollit anim id est laborum.
  Lorem ipsum dolor sit amet, consectetur adipisicing elit, sed do eiusmod tempor incididunt ut labore et dolore magna aliqua. Ut enim ad minim veniam, quis nostrud exercitation ullamco laboris nisi ut aliquip ex ea commodo consequat. Duis aute irure dolor in reprehenderit in voluptate velit esse cillum dolore eu fugiat nulla pariatur. Excepteur sint occaecat cupidatat non proident, sunt in culpa qui officia deserunt mollit anim id est laborum.

  \section{Ejercicio 1}

  \subsection{Aumento del campo de visión}

  \subsection{Detención más rápida cerca del objetivo}
    Para que el robot no pare demasiado lento sobre el punto objetivo, acotamos la mínima velocidad que puede tener el robot, ya que al estar muy cerca del objetivo, el vector de movimiento resultante tiene módulo muy pequeño y por tanto la velocidad del robot es muy baja. Acotando por abajo dicha velocidad conseguimos que el robot no deje nunca de moverse a un ritmo aceptable.

  \subsection{Corrección del comportamiento suicida}
    Para solucionar este comportamiento simplemente hemos acotado la velocidad máxima del robot. Así, aunque el módulo del vector de movimiento resultante (debido a un factor de repulsión muy grande) fuera muy grande, al estar acotado daría tiempo al robot para girar y evitar el obstáculo.

  \subsection{Salida de mínimos locales}
    Lo primero ha sido determinar la siguiente cuestión: ¿Cómo sabemos que estamos atascados en un mínimo local? Nosotros lo determinamos si no salimos de un radio de 0.5 unidades de distancia durante 5 segundos (25 iteraciones a 5Hz). Una vez hemos decidido, mandamos al robot un goal en una posición aleatoria del mapa. Esto lo mantenemos durante 10 segundos, que es un tiempo viable para salir de una esquina, y después de eso volvemos a mandarle el goal original.

  \subsection{Reducción de oscilaciones}
    Para reducir las oscilaciones hemos ideado lo siguiente: como es la componente repulsiva la que las causa debido a las variaciones bruscas de la dirección del vector, nosotros le aplicamos la media a las dos últimas componentes repulsivas calculadas. Así se reduce la diferencia entre las iteraciones anteriores y la actual, evitando oscilaciones.
\end{document}
